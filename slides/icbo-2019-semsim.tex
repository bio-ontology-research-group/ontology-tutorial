\documentclass{beamer}
\usepackage{booktabs}
\usepackage{pdfpages}
\usepackage{mathtools}
\usepackage{enumerate}
\usepackage{multirow,tabularx}
\usepackage{booktabs}
\usepackage{pdfpages}
\usepackage{proof}
\usepackage{cancel}
\usepackage{chronology}
\usepackage{graphicx}
\usepackage{ulem}
\usepackage{amsmath}
\usepackage{amssymb}
\usepackage{color}
\usepackage{animate}

\PassOptionsToPackage{usenames,dvipsnames,svgnames}{xcolor}  
\usepackage{tikz}
\usepackage{tkz-graph}


\usepackage{wasysym}
\usepackage{proof}
\usepackage{cancel}
\usepackage{chronology}
\usepackage{graphicx}
\usepackage{ulem}
\usepackage{amsmath}
\usepackage{amssymb}
\usepackage{color}
\usepackage{xcolor}
\usepackage{soul}
%\usepackage{pstricks}
\setbeamertemplate{navigation symbols}{}

\newcommand{\norm}[1]{\left\lVert#1\right\rVert}
\newcommand{\el}{$\mathcal{EL}^{++}$}
\renewcommand{\Re}{\mathbb{R}}
\newcommand{\BigO}[1]{\ensuremath{\operatorname{O}\bigl(#1\bigr)}}
\newcommand{\myul}[2][blue]{\sethlcolor{#1}\hl{#2}\setulcolor{black}}

\newcommand<>{\cunderline}[3]{\only<#1>{#3}\only<#2>{\underline{#3}}}
\newcommand<>{\cem}[3]{\only<#1>{#3}\only<#2>{\ul{#3}}}
\newcommand<>{\cgray}[3]{\only<#1>{#3}\only<#2>{\textcolor{gray}{#3}}}
\newcommand<>{\colorize}[4]{\only<#1>{#4}\only<#2>{\textcolor{#3}{#4}}}

%\setbeamertemplate{navigation symbols}{}
\addtobeamertemplate{navigation symbols}{}{%
    \usebeamerfont{footline}%
    \usebeamercolor[fg]{footline}%
    \hspace{1em}%
    \insertframenumber/\inserttotalframenumber
}

\renewcommand{\em}{\itshape}

\mode<presentation>
% {
%   \usecolortheme{crane}
% %  \usetheme{Frankfurt}
% }
\mode<presentation>
{
  \usecolortheme{dove}
}

% \mode<presentation>
% {
% \useinnertheme[shadow=true]{rounded}
% \useoutertheme{infolines}
% \usecolortheme{dove}
% \setbeamerfont{block title}{size={}}
% }

\title[Bio-Ontologies]{Machine learning with ontologies}

\author{Robert Hoehndorf}


\date{}

\begin{document}

\begin{frame}
  \titlepage
\end{frame}

\section{Semantic Similarity}

\begin{frame}
  Semantic similarity
  \begin{itemize}
  \item We want to use {\em background knowledge} in ontologies to
    \begin{itemize}
    \item determine similarity between classes,
    \item instances,
    \item and entities with ontology annotations
    \end{itemize}
  \end{itemize}
\end{frame}


\begin{frame}
  \frametitle{How to measure similarity?}
  \begin{itemize}
  \item semantic similarity measures similarity between classes
  \item semantic similarity measures similarity between instances of classes
  \item semantic similarity measures similarity between entities
    {\em annotated} with classes
  \item $\Rightarrow$ reduce all of this to similarity between classes
  \end{itemize}
\end{frame}

\begin{frame}
  \frametitle{How to measure similarity?}
  What properties do we want in a similarity measure?
  \\
  A function $sim: D \times D$ is a similarity on $D$ if, for
  all $x, y \in D$, the function $sim$ is:  \begin{itemize}
    \pause
  \item non-negative: $sim(x,y) \geq 0$ for all $x, y$
    \pause
  \item symmetric: $sim(x,y) = sim(y,x)$
    \pause
  \item reflexive: $sim(x,x) = max_D$
    \pause
    \begin{itemize}
    \item weaker form: $sim(x,x) > sim(x,y)$ for all $x \not= y$
    \end{itemize}
    \pause
  \item $sim(x,x) > sim(x,y)$ for $x\not= y$
    \pause
  \item $sim$ is a {\em normalized} similarity measure if it has
    values in $[0,1]$
  \end{itemize}
\end{frame}

\usetikzlibrary{arrows,positioning,automata}
\begin{frame}
  \frametitle{How to measure similarity?}
  \begin{columns}
    \begin{column}{.6\textwidth}
      {\tiny
        \begin{tikzpicture}[>=stealth',shorten >=1pt,node distance=2cm,on grid,initial/.style    ={}]
          \node[state]          (A)                        {$Thing$};
          \node[state]          (B) [below left =of A]    {$Color$};
          \node[state]          (C) [below right =of A]    {$Shape$};
          \node[state]          (D) [below left =of B]    {$Red$};
          \node[state]          (H) [below right =of B]    {$Green$};
          \node[state]          (E) [below  =of D]    {$Orange$};
          \node[state]          (F) [below =of C]    {$Round$};
          \node[state]          (G) [below right =of C]    {$Square$};
          \tikzset{mystyle/.style={->,double=orange}} 
          \tikzset{every node/.style={fill=white}} 
          \path (B)     edge [mystyle]    node   {$isa$} (A)
          (C)     edge [mystyle]    node   {$isa$} (A) 
          (D)     edge [mystyle]    node   {$isa$} (B)
          (H)     edge [mystyle]    node   {$isa$} (B)
          (E)     edge [mystyle]    node   {$isa$} (D)
          (F)     edge [mystyle]    node   {$isa$} (C)
          (G)     edge [mystyle]    node   {$isa$} (C);
          \tikzset{mystyle/.style={<->,double=orange}}   
          \tikzset{mystyle/.style={<->,relative=true,in=0,out=60,double=orange}}
        \end{tikzpicture}
      }
    \end{column}
    \begin{column}{.4\textwidth}
      \begin{itemize}
        \pause
      \item distance on shortest path (Rada {\em et al.}, 1989)
        \pause
      \item $dist_{Rada}(u,v) = sp(u, isa, v)$
        \pause
      \item $sim_{Rada}(u,v) = \frac{1}{dist_{Rada}(u,v) + 1}$
      \end{itemize}
    \end{column}
  \end{columns}
\end{frame}

\begin{frame}
  \frametitle{How to measure similarity?}
  \begin{columns}
    \begin{column}{.6\textwidth}
      {\tiny
        \begin{tikzpicture}[>=stealth',shorten >=1pt,node distance=2cm,on grid,initial/.style    ={}]
          \node[state]          (A)                        {$Thing$};
          \node[state]          (B) [below left =of A]    {$Color$};
          \node[state]          (C) [below right =of A]    {$Shape$};
          \node[state, fill=green]          (D) [below left =of B]    {$Red$};
          \node[state, fill=green]          (H) [below right =of B]    {$Green$};
          \node[state]          (E) [below  =of D]    {$Orange$};
          \node[state]          (F) [below =of C]    {$Round$};
          \node[state]          (G) [below right =of C]    {$Square$};
          \tikzset{mystyle/.style={->,double=orange}} 
          \tikzset{highlight/.style={->,double=green}} 
          \tikzset{every node/.style={fill=white}}
          \path (B)     edge [mystyle]    node   {$isa$} (A)
          (C)     edge [mystyle]    node   {$isa$} (A) 
          (D)     edge [highlight]    node   {$isa$} (B)
          (H)     edge [highlight]    node   {$isa$} (B)
          (E)     edge [mystyle]    node   {$isa$} (D)
          (F)     edge [mystyle]    node   {$isa$} (C)
          (G)     edge [mystyle]    node   {$isa$} (C);
          \tikzset{mystyle/.style={<->,double=orange}}   
          \tikzset{mystyle/.style={<->,relative=true,in=0,out=60,double=orange}}
        \end{tikzpicture}
      }
    \end{column}
    \begin{column}{.4\textwidth}
      \begin{itemize}
      \item distance on shortest path
        \pause
       \item distance(green, red) = 2
       \item $sim_{Rada}(green, red) = \frac{1}{3}$
      \end{itemize}
    \end{column}
  \end{columns}
\end{frame}

\begin{frame}
  \frametitle{How to measure similarity?}
  \begin{columns}
    \begin{column}{.6\textwidth}
      {\tiny
        \begin{tikzpicture}[>=stealth',shorten >=1pt,node distance=2cm,on grid,initial/.style    ={}]
          \node[state]          (A)                        {$Thing$};
          \node[state]          (B) [below left =of A]    {$Color$};
          \node[state]          (C) [below right =of A]    {$Shape$};
          \node[state]          (D) [below left =of B]    {$Red$};
          \node[state]          (H) [below right =of B]    {$Green$};
          \node[state]          (E) [below  =of D]    {$Orange$};
          \node[state, fill=green]          (F) [below =of C]    {$Round$};
          \node[state, fill=green]          (G) [below right =of C]    {$Square$};
          \tikzset{mystyle/.style={->,double=orange}} 
          \tikzset{highlight/.style={->,double=green}} 
          \tikzset{every node/.style={fill=white}}
          \path (B)     edge [mystyle]    node   {$isa$} (A)
          (C)     edge [mystyle]    node   {$isa$} (A) 
          (D)     edge [mystyle]    node   {$isa$} (B)
          (H)     edge [mystyle]    node   {$isa$} (B)
          (E)     edge [mystyle]    node   {$isa$} (D)
          (F)     edge [highlight]    node   {$isa$} (C)
          (G)     edge [highlight]    node   {$isa$} (C);
          \tikzset{mystyle/.style={<->,double=orange}}   
          \tikzset{mystyle/.style={<->,relative=true,in=0,out=60,double=orange}}
        \end{tikzpicture}
      }
    \end{column}
    \begin{column}{.4\textwidth}
      \begin{itemize}
       \item distance on shortest path
       \item distance(square, round) = 2
       \item $sim_{Rada}(square, round) = \frac{1}{3}$
      \end{itemize}
    \end{column}
  \end{columns}
\end{frame}

\begin{frame}
  \frametitle{How to measure similarity?}
  \begin{columns}
    \begin{column}{.6\textwidth}
      {\tiny
        \begin{tikzpicture}[>=stealth',shorten >=1pt,node distance=2cm,on grid,initial/.style    ={}]
          \node[state]          (A)                        {$Thing$};
          \node[state, fill=green]          (B) [below left =of A]    {$Color$};
          \node[state]          (C) [below right =of A]    {$Shape$};
          \node[state]          (D) [below left =of B]    {$Red$};
          \node[state]          (H) [below right =of B]    {$Green$};
          \node[state, fill=green]          (E) [below  =of D]    {$Orange$};
          \node[state]          (F) [below =of C]    {$Round$};
          \node[state]          (G) [below right =of C]    {$Square$};
          \tikzset{mystyle/.style={->,double=orange}} 
          \tikzset{highlight/.style={->,double=green}} 
          \tikzset{every node/.style={fill=white}}
          \path (B)     edge [mystyle]    node   {$isa$} (A)
          (C)     edge [mystyle]    node   {$isa$} (A) 
          (D)     edge [highlight]    node   {$isa$} (B)
          (H)     edge [mystyle]    node   {$isa$} (B)
          (E)     edge [highlight]    node   {$isa$} (D)
          (F)     edge [mystyle]    node   {$isa$} (C)
          (G)     edge [mystyle]    node   {$isa$} (C);
          \tikzset{mystyle/.style={<->,double=orange}}   
          \tikzset{mystyle/.style={<->,relative=true,in=0,out=60,double=orange}}
        \end{tikzpicture}
      }
    \end{column}
    \begin{column}{.4\textwidth}
      \begin{itemize}
       \item distance on shortest path
       \item distance(orange, color) = 2
       \item $sim_{Rada}(orange, color) = \frac{1}{3}$
      \end{itemize}
    \end{column}
  \end{columns}
\end{frame}

\begin{frame}
  \frametitle{How to measure similarity?}
  \begin{itemize}
  \item shortest path is not always intuitive
    \pause
  \item we need a way to determine {\em specificity} of a class
    \begin{itemize}
    \item number of ancestors
    \item number of children
    \item information content
    \end{itemize}
    \pause
  \item {\em density} of a branch in the ontology
    \begin{itemize}
    \item number of siblings
    \item information content
    \end{itemize}
    \pause
  \item account for different edge types
    \begin{itemize}
    \item non-uniform edge weighting
    \end{itemize}
  \end{itemize}
\end{frame}

\begin{frame}
  \frametitle{How to measure similarity}
  \begin{itemize}
  \item term specificity measure $\sigma: C \mapsto \mathbb{R}$:
    \begin{itemize}
    \item $x \sqsubseteq y \rightarrow \sigma(x) \geq \sigma(y)$
    \end{itemize}
    \pause
  \item intrinsic:
    \begin{itemize}
    \item $\sigma(x) = f(depth(x))$
    \item $\sigma(x) = f(A(x))$ (for ancestors $A(x)$)
    \item $\sigma(x) = f(D(x))$ (for descendants $D(x)$)
    \item many more, e.g., Zhou et al.: $\sigma(x) = k \cdot \Big( 1-\frac{\log
        |D(x)|}{\log |C|} \Big) + (1-k) \frac{\log depth(x)}{\log
        depth(G_T)} $
    \end{itemize}
    \pause
  \item extrinsic:
    \begin{itemize}
    \item $\sigma(x)$ defined as a function of instances (or annotations) $I$
      \begin{itemize}
      \item note: the number of instances monotonically decreases with
        increasing depth in taxonomies
      \end{itemize}
    \item Resnik 1995: $eIC_{Resnik}(x) = -\log p(x)$ (with $p(x) =
      \frac{|I(x)|}{|I|}$)
      \begin{itemize}
      \item in biology, one of the most popular specificity measure when
        annotations are present
      \end{itemize}
    \end{itemize}
  \end{itemize}
\end{frame}

\begin{frame}
  \frametitle{How to measure similarity?}
  \begin{columns}
    \begin{column}{.6\textwidth}
      {\tiny
        \begin{tikzpicture}[>=stealth',shorten >=1pt,node distance=2cm,on grid,initial/.style    ={}]
          \node[state,label=below:$0.0$]          (A)                        {$Thing$};
          \node[state,label=below:$1.0$]          (B) [below left =of A]    {$Color$};
          \node[state,label=right:$1.0$]          (C) [below right =of A]    {$Shape$};
          \node[state,label=right:$2.0$]          (D) [below left =of B]    {$Red$};
          \node[state,label=below:$2.0$]          (H) [below right =of B]    {$Green$};
          \node[state,label=below:$3.0$]          (E) [below  =of D]    {$Orange$};
          \node[state,label=below:$2.0$]          (F) [below =of C]    {$Round$};
          \node[state,label=below:$2.0$]          (G) [below right =of C]    {$Square$};
          \tikzset{mystyle/.style={->,double=orange}} 
          \tikzset{highlight/.style={->,double=green}} 
          \tikzset{every node/.style={fill=white}}
          \path (B)     edge [mystyle]    node   {$isa$} (A)
          (C)     edge [mystyle]    node   {$isa$} (A) 
          (D)     edge [mystyle]    node   {$isa$} (B)
          (H)     edge [mystyle]    node   {$isa$} (B)
          (E)     edge [mystyle]    node   {$isa$} (D)
          (F)     edge [mystyle]    node   {$isa$} (C)
          (G)     edge [mystyle]    node   {$isa$} (C);
          \tikzset{mystyle/.style={<->,double=orange}}   
          \tikzset{mystyle/.style={<->,relative=true,in=0,out=60,double=orange}}
        \end{tikzpicture}
      }
    \end{column}
    \begin{column}{.4\textwidth}
      \begin{itemize}
      \item Resnik 1995: similarity between $x$ and $y$ is the
        information content of the {\em most informative common
          ancestor}
      \end{itemize}
    \end{column}
  \end{columns}
\end{frame}

\begin{frame}
  \frametitle{How to measure similarity?}
  \begin{columns}
    \begin{column}{.6\textwidth}
      {\tiny
        \begin{tikzpicture}[>=stealth',shorten >=1pt,node distance=2cm,on grid,initial/.style    ={}]
          \node[state,label=below:$0.0$]          (A)                        {$Thing$};
          \node[state,label=below:$1.0$]          (B) [below left =of A]    {$Color$};
          \node[state,label=right:$1.0$]          (C) [below right =of A]    {$Shape$};
          \node[state,fill=green,label=right:$2.0$]          (D) [below left =of B]    {$Red$};
          \node[state,fill=green,label=below:$2.0$]          (H) [below right =of B]    {$Green$};
          \node[state,label=below:$3.0$]          (E) [below  =of D]    {$Orange$};
          \node[state,label=below:$2.0$]          (F) [below =of C]    {$Round$};
          \node[state,label=below:$2.0$]          (G) [below right =of C]    {$Square$};
          \tikzset{mystyle/.style={->,double=orange}} 
          \tikzset{highlight/.style={->,double=green}} 
          \tikzset{every node/.style={fill=white}}
          \path (B)     edge [mystyle]    node   {$isa$} (A)
          (C)     edge [mystyle]    node   {$isa$} (A) 
          (D)     edge [mystyle]    node   {$isa$} (B)
          (H)     edge [mystyle]    node   {$isa$} (B)
          (E)     edge [mystyle]    node   {$isa$} (D)
          (F)     edge [mystyle]    node   {$isa$} (C)
          (G)     edge [mystyle]    node   {$isa$} (C);
          \tikzset{mystyle/.style={<->,double=orange}}   
          \tikzset{mystyle/.style={<->,relative=true,in=0,out=60,double=orange}}
        \end{tikzpicture}
      }
    \end{column}
    \begin{column}{.4\textwidth}
      \begin{itemize}
      \item Resnik 1995: similarity between $x$ and $y$ is the
        information content of the {\em most informative common
          ancestor}
      \end{itemize}
    \end{column}
  \end{columns}
\end{frame}

\begin{frame}
  \frametitle{How to measure similarity?}
  \begin{columns}
    \begin{column}{.6\textwidth}
      {\tiny
        \begin{tikzpicture}[>=stealth',shorten >=1pt,node distance=2cm,on grid,initial/.style    ={}]
          \node[state,label=below:$0.0$]          (A)                        {$Thing$};
          \node[state,fill=yellow,label=below:$1.0$]          (B) [below left =of A]    {$Color$};
          \node[state,label=right:$1.0$]          (C) [below right =of A]    {$Shape$};
          \node[state,fill=green,label=right:$2.0$]          (D) [below left =of B]    {$Red$};
          \node[state,fill=green,label=below:$2.0$]          (H) [below right =of B]    {$Green$};
          \node[state,label=below:$3.0$]          (E) [below  =of D]    {$Orange$};
          \node[state,label=below:$2.0$]          (F) [below =of C]    {$Round$};
          \node[state,label=below:$2.0$]          (G) [below right =of C]    {$Square$};
          \tikzset{mystyle/.style={->,double=orange}} 
          \tikzset{highlight/.style={->,double=green}} 
          \tikzset{every node/.style={fill=white}}
          \path (B)     edge [mystyle]    node   {$isa$} (A)
          (C)     edge [mystyle]    node   {$isa$} (A) 
          (D)     edge [mystyle]    node   {$isa$} (B)
          (H)     edge [mystyle]    node   {$isa$} (B)
          (E)     edge [mystyle]    node   {$isa$} (D)
          (F)     edge [mystyle]    node   {$isa$} (C)
          (G)     edge [mystyle]    node   {$isa$} (C);
          \tikzset{mystyle/.style={<->,double=orange}}   
          \tikzset{mystyle/.style={<->,relative=true,in=0,out=60,double=orange}}
        \end{tikzpicture}
      }
    \end{column}
    \begin{column}{.4\textwidth}
      \begin{itemize}
      \item Resnik 1995: similarity between $x$ and $y$ is the
        information content of the {\em most informative common
          ancestor}
      \end{itemize}
    \end{column}
  \end{columns}
\end{frame}

\begin{frame}
  \frametitle{How to measure similarity?}
  \begin{columns}
    \begin{column}{.6\textwidth}
      {\tiny
        \begin{tikzpicture}[>=stealth',shorten >=1pt,node distance=2cm,on grid,initial/.style    ={}]
          \node[state,label=below:$0.0$]          (A)                        {$Thing$};
          \node[state,fill=yellow,label=below:$1.0$]          (B) [below left =of A]    {$Color$};
          \node[state,label=right:$1.0$]          (C) [below right =of A]    {$Shape$};
          \node[state,fill=green,label=right:$2.0$]          (D) [below left =of B]    {$Red$};
          \node[state,fill=green,label=below:$2.0$]          (H) [below right =of B]    {$Green$};
          \node[state,label=below:$3.0$]          (E) [below  =of D]    {$Orange$};
          \node[state,label=below:$2.0$]          (F) [below =of C]    {$Round$};
          \node[state,label=below:$2.0$]          (G) [below right =of C]    {$Square$};
          \tikzset{mystyle/.style={->,double=orange}} 
          \tikzset{highlight/.style={->,double=green}} 
          \tikzset{every node/.style={fill=white}}
          \path (B)     edge [mystyle]    node   {$isa$} (A)
          (C)     edge [mystyle]    node   {$isa$} (A) 
          (D)     edge [mystyle]    node   {$isa$} (B)
          (H)     edge [mystyle]    node   {$isa$} (B)
          (E)     edge [mystyle]    node   {$isa$} (D)
          (F)     edge [mystyle]    node   {$isa$} (C)
          (G)     edge [mystyle]    node   {$isa$} (C);
          \tikzset{mystyle/.style={<->,double=orange}}   
          \tikzset{mystyle/.style={<->,relative=true,in=0,out=60,double=orange}}
        \end{tikzpicture}
      }
    \end{column}
    \begin{column}{.4\textwidth}
      \begin{itemize}
      \item Resnik 1995: similarity between $x$ and $y$ is the
        information content of the {\em most informative common
          ancestor}
        \item $sim_{Resnik}(Green, Red) = 1.0$
      \end{itemize}
    \end{column}
  \end{columns}
\end{frame}

\begin{frame}
  \frametitle{How to measure similarity?}
  \begin{columns}
    \begin{column}{.6\textwidth}
      {\tiny
        \begin{tikzpicture}[>=stealth',shorten >=1pt,node distance=2cm,on grid,initial/.style    ={}]
          \node[state,label=below:$0.0$]          (A)                        {$Thing$};
          \node[state,fill=yellow,label=below:$1.0$]          (B) [below left =of A]    {$Color$};
          \node[state,label=right:$1.0$]          (C) [below right =of A]    {$Shape$};
          \node[state,label=right:$2.0$]          (D) [below left =of B]    {$Red$};
          \node[state,fill=green,label=below:$2.0$]          (H) [below right =of B]    {$Green$};
          \node[state,fill=green,label=below:$3.0$]          (E) [below  =of D]    {$Orange$};
          \node[state,label=below:$2.0$]          (F) [below =of C]    {$Round$};
          \node[state,label=below:$2.0$]          (G) [below right =of C]    {$Square$};
          \tikzset{mystyle/.style={->,double=orange}} 
          \tikzset{highlight/.style={->,double=green}} 
          \tikzset{every node/.style={fill=white}}
          \path (B)     edge [mystyle]    node   {$isa$} (A)
          (C)     edge [mystyle]    node   {$isa$} (A) 
          (D)     edge [mystyle]    node   {$isa$} (B)
          (H)     edge [mystyle]    node   {$isa$} (B)
          (E)     edge [mystyle]    node   {$isa$} (D)
          (F)     edge [mystyle]    node   {$isa$} (C)
          (G)     edge [mystyle]    node   {$isa$} (C);
          \tikzset{mystyle/.style={<->,double=orange}}   
          \tikzset{mystyle/.style={<->,relative=true,in=0,out=60,double=orange}}
        \end{tikzpicture}
      }
    \end{column}
    \begin{column}{.4\textwidth}
      \begin{itemize}
      \item Resnik 1995: similarity between $x$ and $y$ is the
        information content of the {\em most informative common
          ancestor}
        \item $sim_{Resnik}(Green, Orange) = 1.0$
      \end{itemize}
    \end{column}
  \end{columns}
\end{frame}

\begin{frame}
  \frametitle{How to measure similarity?}
  \begin{columns}
    \begin{column}{.6\textwidth}
      {\tiny
        \begin{tikzpicture}[>=stealth',shorten >=1pt,node distance=2cm,on grid,initial/.style    ={}]
          \node[state,fill=yellow,label=below:$0.0$]          (A)                        {$Thing$};
          \node[state,label=below:$1.0$]          (B) [below left =of A]    {$Color$};
          \node[state,label=right:$1.0$]          (C) [below right =of A]    {$Shape$};
          \node[state,label=right:$2.0$]          (D) [below left =of B]    {$Red$};
          \node[state,fill=green,label=below:$2.0$]          (H) [below right =of B]    {$Green$};
          \node[state,label=below:$3.0$]          (E) [below  =of D]    {$Orange$};
          \node[state,label=below:$2.0$]          (F) [below =of C]    {$Round$};
          \node[state,fill=green,label=below:$2.0$]          (G) [below right =of C]    {$Square$};
          \tikzset{mystyle/.style={->,double=orange}} 
          \tikzset{highlight/.style={->,double=green}} 
          \tikzset{every node/.style={fill=white}}
          \path (B)     edge [mystyle]    node   {$isa$} (A)
          (C)     edge [mystyle]    node   {$isa$} (A) 
          (D)     edge [mystyle]    node   {$isa$} (B)
          (H)     edge [mystyle]    node   {$isa$} (B)
          (E)     edge [mystyle]    node   {$isa$} (D)
          (F)     edge [mystyle]    node   {$isa$} (C)
          (G)     edge [mystyle]    node   {$isa$} (C);
          \tikzset{mystyle/.style={<->,double=orange}}   
          \tikzset{mystyle/.style={<->,relative=true,in=0,out=60,double=orange}}
        \end{tikzpicture}
      }
    \end{column}
    \begin{column}{.4\textwidth}
      \begin{itemize}
      \item Resnik 1995: similarity between $x$ and $y$ is the
        information content of the {\em most informative common
          ancestor}
        \item $sim_{Resnik}(Square, Orange) = 0.0$
      \end{itemize}
    \end{column}
  \end{columns}
\end{frame}

\begin{frame}
  \frametitle{How to measure similarity?}
  \begin{itemize}
  \item (Red, Green) and (Orange, Green) have the same similarity
  \item need to incorporate the specificity of the compared classes
  \end{itemize}
\end{frame}

\begin{frame}
  \frametitle{How to measure similarity?}
  \begin{columns}
    \begin{column}{.6\textwidth}
      {\tiny
        \begin{tikzpicture}[>=stealth',shorten >=1pt,node distance=2cm,on grid,initial/.style    ={}]
          \node[state,label=below:$0.0$]          (A)                        {$Thing$};
          \node[state,fill=yellow,label=below:$1.0$]          (B) [below left =of A]    {$Color$};
          \node[state,label=right:$1.0$]          (C) [below right =of A]    {$Shape$};
          \node[state,fill=green,label=right:$2.0$]          (D) [below left =of B]    {$Red$};
          \node[state,fill=green,label=below:$2.0$]          (H) [below right =of B]    {$Green$};
          \node[state,label=below:$3.0$]          (E) [below  =of D]    {$Orange$};
          \node[state,label=below:$2.0$]          (F) [below =of C]    {$Round$};
          \node[state,label=below:$2.0$]          (G) [below right =of C]    {$Square$};
          \tikzset{mystyle/.style={->,double=orange}} 
          \tikzset{highlight/.style={->,double=green}} 
          \tikzset{every node/.style={fill=white}}
          \path (B)     edge [mystyle]    node   {$isa$} (A)
          (C)     edge [mystyle]    node   {$isa$} (A) 
          (D)     edge [mystyle]    node   {$isa$} (B)
          (H)     edge [mystyle]    node   {$isa$} (B)
          (E)     edge [mystyle]    node   {$isa$} (D)
          (F)     edge [mystyle]    node   {$isa$} (C)
          (G)     edge [mystyle]    node   {$isa$} (C);
          \tikzset{mystyle/.style={<->,double=orange}}   
          \tikzset{mystyle/.style={<->,relative=true,in=0,out=60,double=orange}}
        \end{tikzpicture}
      }
    \end{column}
    \begin{column}{.4\textwidth}
      \begin{itemize}
      \item Lin 1998: $sim_{Lin}(x,y) = \frac{2\cdot
          IC(MICA(x,y))}{IC(x) + IC(y)}$
        \pause
      \item $sim_{Lin}(Green, Red) = 0.5$
      \end{itemize}
    \end{column}
  \end{columns}
\end{frame}

\begin{frame}
  \frametitle{How to measure similarity?}
  \begin{columns}
    \begin{column}{.6\textwidth}
      {\tiny
        \begin{tikzpicture}[>=stealth',shorten >=1pt,node distance=2cm,on grid,initial/.style    ={}]
          \node[state,label=below:$0.0$]          (A)                        {$Thing$};
          \node[state,fill=yellow,label=below:$1.0$]          (B) [below left =of A]    {$Color$};
          \node[state,label=right:$1.0$]          (C) [below right =of A]    {$Shape$};
          \node[state,label=right:$2.0$]          (D) [below left =of B]    {$Red$};
          \node[state,fill=green,label=below:$2.0$]          (H) [below right =of B]    {$Green$};
          \node[state,fill=green,label=below:$3.0$]          (E) [below  =of D]    {$Orange$};
          \node[state,label=below:$2.0$]          (F) [below =of C]    {$Round$};
          \node[state,label=below:$2.0$]          (G) [below right =of C]    {$Square$};
          \tikzset{mystyle/.style={->,double=orange}} 
          \tikzset{highlight/.style={->,double=green}} 
          \tikzset{every node/.style={fill=white}}
          \path (B)     edge [mystyle]    node   {$isa$} (A)
          (C)     edge [mystyle]    node   {$isa$} (A) 
          (D)     edge [mystyle]    node   {$isa$} (B)
          (H)     edge [mystyle]    node   {$isa$} (B)
          (E)     edge [mystyle]    node   {$isa$} (D)
          (F)     edge [mystyle]    node   {$isa$} (C)
          (G)     edge [mystyle]    node   {$isa$} (C);
          \tikzset{mystyle/.style={<->,double=orange}}   
          \tikzset{mystyle/.style={<->,relative=true,in=0,out=60,double=orange}}
        \end{tikzpicture}
      }
    \end{column}
    \begin{column}{.4\textwidth}
      \begin{itemize}
      \item Lin 1998: $sim_{Lin}(x,y) = \frac{2\cdot
          IC(MICA(x,y))}{IC(x) + IC(y)}$
      \item $sim_{Lin}(Green, Orange) = 0.4$
      \end{itemize}
    \end{column}
  \end{columns}
\end{frame}

\begin{frame}
  \frametitle{How to measure similarity?}
  \begin{itemize}
  \item many(!) others:
    \begin{itemize}
    \item Jiang \& Conrath 1997
    \item Mazandu \& Mulder 2013
    \item Schlicker et al. 2009
    \item ...
  \end{itemize}
  \end{itemize}
\end{frame}

\begin{frame}
  \frametitle{How to measure similarity?}
  \begin{itemize}
  \item we only looked at comparing pairs of classes
  \item mostly, we want to compare {\em sets} of classes
    \begin{itemize}
    \item set of GO annotations
    \item set of signs and symptoms
    \item set of phenotypes
    \end{itemize}
  \item two approaches:
    \begin{itemize}
    \item compare each class individually, then merge
    \item directly set-based similarity measures
    \end{itemize}
  \end{itemize}
\end{frame}

\begin{frame}
  \frametitle{How to measure similarity?}
  \begin{columns}
    \begin{column}{.6\textwidth}
      {\tiny
        \begin{tikzpicture}[>=stealth',shorten >=1pt,node distance=2cm,on grid,initial/.style    ={}]
          \node[state,label=below:$0.0$]          (A)                        {$Thing$};
          \node[state,label=below:$1.0$]          (B) [below left =of A]    {$Color$};
          \node[state,label=right:$1.0$]          (C) [below right =of A]    {$Shape$};
          \node[state,fill=gray,label=right:$2.0$]          (D) [below left =of B]    {$Red$};
          \node[state,label=below:$2.0$]          (H) [below right =of B]    {$Green$};
          \node[state,fill=green,label=below:$3.0$]          (E) [below  =of D]    {$Orange$};
          \node[state,fill=gray,label=below:$2.0$]          (F) [below =of C]    {$Round$};
          \node[state,fill=green,label=below:$2.0$]          (G) [below right =of C]    {$Square$};
          \tikzset{mystyle/.style={->,double=orange}} 
          \tikzset{highlight/.style={->,double=green}} 
          \tikzset{every node/.style={fill=white}}
          \path (B)     edge [mystyle]    node   {$isa$} (A)
          (C)     edge [mystyle]    node   {$isa$} (A) 
          (D)     edge [mystyle]    node   {$isa$} (B)
          (H)     edge [mystyle]    node   {$isa$} (B)
          (E)     edge [mystyle]    node   {$isa$} (D)
          (F)     edge [mystyle]    node   {$isa$} (C)
          (G)     edge [mystyle]    node   {$isa$} (C);
          \tikzset{mystyle/.style={<->,double=orange}}   
          \tikzset{mystyle/.style={<->,relative=true,in=0,out=60,double=orange}}
        \end{tikzpicture}
      }
    \end{column}
    \begin{column}{.4\textwidth}
      \begin{itemize}
      \item similarity between a square-and-orange thing and a
        round-and-red thing
        \pause
      \item Pesquita et al., 2007:
        $simGIC(X,Y) = \frac{\sum_{c \in A(X) \cap A(Y)}
          IC(c)}{\sum_{c \in A(X) \cup A(Y)} IC(c)}$
      \end{itemize}
    \end{column}
  \end{columns}
\end{frame}

\begin{frame}
  \frametitle{How to measure similarity?}
  \begin{columns}
    \begin{column}{.6\textwidth}
      {\tiny
        \begin{tikzpicture}[>=stealth',shorten >=1pt,node distance=2cm,on grid,initial/.style    ={}]
          \node[state,fill=pink,label=below:$0.0$]          (A)                        {$Thing$};
          \node[state,fill=pink,label=below:$1.0$]          (B) [below left =of A]    {$Color$};
          \node[state,fill=pink,label=right:$1.0$]          (C) [below right =of A]    {$Shape$};
          \node[state,fill=gray,label=right:$2.0$]          (D) [below left =of B]    {$Red$};
          \node[state,label=below:$2.0$]          (H) [below right =of B]    {$Green$};
          \node[state,fill=green,label=below:$3.0$]          (E) [below  =of D]    {$Orange$};
          \node[state,fill=gray,label=below:$2.0$]          (F) [below =of C]    {$Round$};
          \node[state,fill=green,label=below:$2.0$]          (G) [below right =of C]    {$Square$};
          \tikzset{mystyle/.style={->,double=orange}} 
          \tikzset{highlight/.style={->,double=green}} 
          \tikzset{every node/.style={fill=white}}
          \path (B)     edge [mystyle]    node   {$isa$} (A)
          (C)     edge [mystyle]    node   {$isa$} (A) 
          (D)     edge [mystyle]    node   {$isa$} (B)
          (H)     edge [mystyle]    node   {$isa$} (B)
          (E)     edge [mystyle]    node   {$isa$} (D)
          (F)     edge [mystyle]    node   {$isa$} (C)
          (G)     edge [mystyle]    node   {$isa$} (C);
          \tikzset{mystyle/.style={<->,double=orange}}   
          \tikzset{mystyle/.style={<->,relative=true,in=0,out=60,double=orange}}
        \end{tikzpicture}
      }
    \end{column}
    \begin{column}{.4\textwidth}
      \begin{itemize}
      \item similarity between a square-and-orange thing and a
        round-and-red thing
      \item Pesquita et al., 2007:
        $simGIC(X,Y) = \frac{\sum_{c \in A(X) \cap A(Y)}
          IC(c)}{\sum_{c \in A(X) \cup A(Y)} IC(c)}$
      \item $simGIC(so,rr) = \frac{2}{11}$
      \end{itemize}
    \end{column}
  \end{columns}
\end{frame}

\begin{frame}
  \frametitle{How to measure similarity?}
  \begin{itemize}
  \item alternatively: use different merging strategies
  \item common: average, maximum, {\bf best-matching average}
    \begin{itemize}
    \item Average: $sim_A(X,Y) = \frac{\sum_{x\in X} \sum_{y \in Y} sim(x,y)}{|X| \times |Y|}$
    \item Max average: $sim_{MA}(X,Y) = \frac{1}{|X|} \sum_{x\in X} \max_{y \in Y} sim(x,y)$
    \item Best match average: $sim_{BMA}(X,Y) = \frac{sim_{MA}(X,Y) + sim_{MA}(Y,X)}{2}$
    \end{itemize}
  \end{itemize}
\end{frame}

\begin{frame}
  \frametitle{How to measure similarity?}
  \begin{itemize}
  \item Semantic Measures Library:
    \begin{itemize}
    \item comprehensive Java library
    \item \url{http://www.semantic-measures-library.org/}
    \end{itemize}
  \item R packages: GOSim, GOSemSim, HPOSim, LSAfun,
    ontologySimilarity,...
  \item Python: sematch, fastsemsim (GO only)
  \end{itemize}
\end{frame}

% \begin{frame}
%   \frametitle{Applications of semantic similarity}
%   \begin{itemize}
%   \item ontologies are used {\em almost everywhere} in biology
%   \item many applications of semantic similarity:
%     \begin{itemize}
%     \item predicting interacting proteins
%     \item predict candidate genes
%       \begin{itemize}
%       \item using the guilt-by-association principle, or without
%       \end{itemize}
%     \item predict drug targets and indications
%     \item as features in machine learning models
%     \end{itemize}
%   \end{itemize}
% \end{frame}

\begin{frame}
  \frametitle{Applications of semantic similarity}
  \begin{block}{Hypothesis}
    Interacting proteins have similar functions.
  \end{block}
  \begin{itemize}
  \item relies on background knowledge about functions (encoded in GO)
  \item ``similarity'' can mean:
    \begin{itemize}
    \item part of the same pathway
    \item siblings of a common super-class
    \item located in the same location
    \end{itemize}
  \item set-based comparison of GO functions
    \begin{itemize}
    \item single GO hierarchy or all?
    \item which similarity measure?
    \end{itemize}
  \end{itemize}
\end{frame}

\begin{frame}
  \frametitle{Applications of semantic similarity}
  \centerline{\includegraphics[width=.8\textwidth]{ppi1.png}}
\end{frame}

\begin{frame}
  \frametitle{Applications of semantic similarity}
  \centerline{\includegraphics[width=.8\textwidth]{ppi2.png}}
\end{frame}

\begin{frame}
  \frametitle{Applications of semantic similarity}
  \centerline{\includegraphics[width=.8\textwidth]{ppi3.png}}
\end{frame}

\begin{frame}
  \frametitle{Applications of semantic similarity}
  \begin{itemize}
  \item no obvious choice of similarity measure
  \item depends on application
    \begin{itemize}
    \item example: predicting PPIs in different organisms may benefit from a
      different similarity measure!
    \end{itemize}
  \item different similarity measures may react differently to biases
    in data
  \item needs some testing and experience
  \end{itemize}
\end{frame}

\begin{frame}
  \frametitle{Applications of semantic similarity}
  Recommendations:
  \begin{itemize}
  \item use Resnik's information content measure
  \item use Resnik's similarity
  \item use Best Match Average
  \item use the full ontology
  \item classify your ontology using an automated reasoner before
    applying semantic similarity
    \begin{itemize}
    \item although many ontologies come pre-classified
    \end{itemize}
  \item $\Rightarrow$ but there are many exceptions
    \begin{itemize}
    \item similar location $\Rightarrow$ use location subset of GO
    \item developmental phenotypes $\Rightarrow$ use developmental
      branch of phenotype ontology
    \end{itemize}
  \end{itemize}
\end{frame}

\begin{frame}
  \frametitle{Applications of semantic similarity}
  \begin{itemize}
  \item choice of ontology determines the kind of similarity
  \item functional similarity: Gene Ontology
  \item anatomical, structural similarity: anatomy ontologies (Uberon,
    MA, FMA, etc.)
  \item phenotypic similarity: phenotype ontology (HPO, MP, etc.)
  \item chemical structural similarity: ChEBI
  \end{itemize}
\end{frame}

% \begin{frame}
%   \frametitle{Applications of semantic similarity}
%   \begin{itemize}
%   \item phenotypic similarity used to:
%     \begin{itemize}
%     \item diagnosis: similarity between patient phenotypes and disease
%       phenotypes
%       \begin{itemize}
%       \item also between patient phenotypes and gene--phenotype
%         associations
%       \item Phenomizer: \url{http://compbio.charite.de/phenomizer/}
%       \end{itemize}
%     \item disease modules: similarity between disease and disease
%     \item clustering/stratification: similarity between patient and patient
%     \item disease gene discovery: similarity between patient/disease
%       phenotypes and gene--phenotype associations
%       \begin{itemize}
%       \item in humans
%       \item in model organisms
%       \end{itemize}
%     \item drug repurposing: side-effect similarity; similarity between
%       side effect profile and gene--disease associations
%     \end{itemize}
%   \end{itemize}
% \end{frame}

% \begin{frame}
%   \frametitle{Applications of semantic similarity}
%   \begin{itemize}
%   \item comparing entities annotated with {\em different}
%     ontologies/vocabularies of the {\em same} (or related) domains
%     \begin{itemize}
%     \item medical: UMLS, HPO, DO, ORDO, NCIT, ICD, SNOMED CT, MeSH, ...
%     \item phenotype: HPO, MP, CPO, WBPhenotype, FBCV, MeSH, ...
%     \item chemical: ChEBI, MeSH, DrOn, RXNorm, DrugBank, ...
%     \end{itemize}
%   \item needs mapping, alignment, or integration
%     \begin{itemize}
%     \item mapping: given a term $t$, find corresponding class in
%       ontology $O$
%       \begin{itemize}
%       \item can be 1:1, 1:n, n:1, n:m
%       \item $t$ can be from ontology, vocabulary, database, or text
%       \item use $O$ for analysis
%       \end{itemize}
%     \item alignment: given two ontologies or vocabularies $O_1$ and
%       $O_2$, find all mappings between classes/terms in $O_1$ and $O_2$
%       \begin{itemize}
%       \item applicable to ontologies and vocabularies
%       \item use $O_1$ or $O_2$ for analysis
%       \end{itemize}
%     \item integration: given two ontologies $O_1$ and $O_2$, combine
%       both ontologies into a single ontology $O$
%       \begin{itemize}
%       \item maintain meaning of classes
%       \item use $O$ for analysis
%       \end{itemize}
%     \end{itemize}
%   \end{itemize}
% \end{frame}

% \begin{frame}
%   \frametitle{Applications of semantic similarity}
%   \begin{itemize}
%   \item lexical mappings: use class labels (and synonyms) to find matches
%     \begin{itemize}
%     \item hypertension ({\tt HP:0000822}) and hypertension ({\tt MP:0000231})
%     \end{itemize}
%   \item semantic mappings: use class axioms to find matches
%     \begin{itemize}
%     \item pulmonary valve stenosis ({\tt MP:0006182}) and Pulmonic
%       stenosis ({\tt HP:0001642})
%     \item both definitions based on constricted ({\tt PATO:0001847})
%       and pulmonary valve ({\tt UBERON:0002146})
%     \end{itemize}
%   \item hybrid: combine lexical and semantic mappings
%   \end{itemize}
% \end{frame}

% \begin{frame}
%   \frametitle{Applications of semantic similarity}
%   tools for ontology mapping, matching, integration:
%   \begin{itemize}
%   \item AgreementMaker Light: \url{https://github.com/AgreementMakerLight/AML-Jar}
%     \begin{itemize}
%     \item structural (semantic) and lexical matches
%     \item can use domain-specific background knowledge
%     \end{itemize}
%   \item LogMap: \url{https://github.com/ernestojimenezruiz/logmap-matcher}
%     \begin{itemize}
%     \item structural (semantic) and lexical matches
%     \item biology-themed versions
%     \end{itemize}
%   \item NCBO Annotator: \url{https://bioportal.bioontology.org/annotator}
%     \begin{itemize}
%     \item lexical matches only
%     \item can annotate full text
%     \end{itemize}
%   \item recent tools and comprehensive ongoing evaluation:
%     \begin{itemize}
%     \item OAEI: \url{http://oaei.ontologymatching.org/}
%     \end{itemize}
%   \end{itemize}
% \end{frame}

% \begin{frame}
%   \frametitle{How to measure similarity?}
%   Recommended reading:
%   \begin{itemize}
%   \item recommended, comprehensive overview: Sebastian Harispe et
%     al. Semantic Similarity from Natural Language and Ontology
%     Analysis. Morgan \& Claypool Publishers, 2015
%   \item Catia Pesquita et al. Semantic Similarity in Biomedical
%     Ontologies. PLoS CB, 2009.
%   \item Maximilian Nickel et al. A Review of Relational Machine
%     Learning for Knowledge Graphs, Proceedings of the IEEE, 2016.
%   \end{itemize}
% \end{frame}

% \begin{frame}
%   \frametitle{How to measure similarity: Quiz}
%   \begin{itemize}
%   \item How many semantic similarity measures are there?
%     \begin{enumerate}[a]
%     \item One (and it is called The Semantic Similarity Measure)
%     \item Three (graph-based, set-based, feature-based)
%     \item Many (depending on context, many functions can determine similarity)
%     \end{enumerate}
%     \pause
%   \item Specificity of an ontology class
%     \begin{enumerate}[a]
%     \item depends on the number of children and ancestors, and the
%       depth
%     \item depends on the number of instances (or annotations)
%     \item can improve similarity estimates significantly
%     \end{enumerate}
%     \pause
%   \item In the presence of (relations between) instances, semantic
%     similarity
%     \begin{enumerate}[a]
%     \item cannot be computed, it only works with ontologies
%     \item can be estimated using only class specificity measures
%     \item can be computed using knowledge graph embeddings
%     \end{enumerate}
%   \end{itemize}
% \end{frame}

% \begin{frame}
%   \frametitle{Hands-on part 2}
%   \begin{itemize}
%   \item back to Jupyter...
%   \item run the ``Semantic Similarity'' part
%     \begin{itemize}
%     \item then find the mouse genotype with the most similar set of
%       phenotypes to ``Tetralogy of Fallot'' (OMIM:187500)
%     \item or: use the data from
%       \url{https://hpo.jax.org/app/download/annotation} to add more
%       diseases and query by disease (hint: a disease is really just a
%       set of phenotypes)
%     \end{itemize}
%   \end{itemize}
% \end{frame}

\end{document}

%%% Local Variables:
%%% mode: latex
%%% TeX-master: "icbo-2019-semsim"
%%% End:
